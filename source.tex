\documentclass{beamer}

\usepackage[utf8]{inputenc}
\usepackage[T2A]{fontenc}
\usepackage[russian]{babel}

\usefonttheme{serif}

\usepackage{amsmath}
\usepackage{amssymb}
\usepackage{amsthm}
\usepackage{amsfonts}
\usepackage{braket}
\usepackage{color}

\newcommand{\pdfimg}[2]{\newline\centerline{\includegraphics[width=#2]{#1}}\newline}

\definecolor{myred}{rgb}{0.75, 0.0, 0.0}
\definecolor{mygreen}{rgb}{0.0, 0.6, 0.1}
\newcommand{\rc}[1]{\textcolor{myred}{#1}}
\newcommand{\gc}[1]{\textcolor{mygreen}{#1}}
\newcommand{\bc}[1]{\textcolor{blue}{#1}}

\begin{document}

\title{Квантовые вычисления}   

\section{Основы}
\frame{
	\frametitle{Кубит}
	$$\alpha\ket{0} + \beta\ket{1}$$
	$$\alpha, \beta \in \mathbb{C};\; |\alpha|^2 + |\beta|^2 = 1$$
	$$\ket{\phi} = \rc{\alpha_\phi} \ket{0} + \rc{\beta_\phi} \ket{1}$$
	$$\ket{\psi} = \bc{\alpha_\psi} \ket{0} + \bc{\beta_\psi} \ket{1}$$
	$$\ket{\phi\psi} = \rc{\alpha_\phi} \bc{\alpha_\psi} \ket{00} + \rc{\alpha_\phi} \bc{\beta_\psi} \ket{01} + 
	  \rc{\beta_\phi} \bc{\alpha_\psi} \ket{10} + \rc{\beta_\phi} \bc{\beta_\psi} \ket{11}$$
	$$\ket{\zeta\eta} = a \ket{00} + b \ket{01} + c \ket{10} + d \ket{11}, bc = ad$$
	$$\ket{\nu_0 \nu_1 ... \nu_n} = \sum_{i={00..0}}^{11..1}{\alpha_i \ket{i}}, \; \sum_{i={00..0}}^{11..1}{|\alpha_i|^2} = 1$$
}

\section{Операторы}
\frame{
	\frametitle{Операторы}
	$$\hat{\rho} = \ket{\phi}\bra{\phi} = (\alpha\ket{0} + \beta\ket{1})(\alpha^\ast\bra{0} + \beta^\ast\bra{1}) =$$
	$$= |\alpha|^2\ket{0}\bra{0} + \gc{\alpha\beta^\ast}\ket{0}\bra{1} + \gc{\beta\alpha^\ast}\ket{1}\bra{0} + |\beta|^2\ket{1}\bra{1}$$
	$$\langle\gc{\alpha^\ast\beta}\rangle = 0 \; \Rightarrow \; \hat{\rho} = |\alpha|^2\ket{0}\bra{0} + |\beta|^2\ket{1}\bra{1}$$
	$$\hat{U}_{not}\ket{0} = \ket{1},\; \hat{U}_{not}\ket{1} = \ket{0}$$
	$$\hat{H}\ket{0} = \frac{\ket{0} + \ket{1}}{\sqrt{2}},\; \hat{H}\ket{1} = \frac{\ket{0} - \ket{1}}{\sqrt{2}}$$
	$$\hat{U}_{cnot}\ket{a,b} = \ket{a, a + b}$$
	$$\hat{U}_{cnot}(\alpha_1\ket{0}_1 + \beta_1\ket{1}_1)(\alpha_2\ket{0}_2 + \beta_2\ket{1}_2)$$
}

\section{Алгоритмы}
\frame{
	\frametitle{Задача Дойча}
	Пусть у нас есть булева функция $f(\bc{x_1, ..., x_n}) \rightarrow \{0, 1\}$, и она:\newline
	либо постоянна --- $\forall x:\: f(x) = C$,\newline
	либо сбалансирована --- $\sum_{i=0}^{2^n-1}{(-1)^{f(i)}} = 0$.
	$$\hat{U}_f\ket{\bc{x_1, ..., x_n},\; \rc{x_{n+1}}} = \ket{x_1, ..., x_n,\; x_{n+1} + f(x_1, ..., x_n)}$$
	\pdfimg{deutsch.pdf}{200pt}
	$$\ket{\theta} = \hat{H}^{(\bc{1..n})} (\hat{U}_{f} \hat{H}^{(\bc{1..n},\rc{n+1})} \ket{\bc{0..0}\rc{1}})_{\bc{1..n}}$$
}
\frame{
	\frametitle{Задача Дойча}
	$$\ket{\phi} = \hat{H}^{(\bc{1..n},\rc{n+1})}\ket{\bc{0..0}\rc{1}} = 
	  \frac{1}{C_1}\bc{(\prod_{i=1}^{n}{\ket{0}_i + \ket{1}_i})}\rc{(\ket{0}_{n+1} - \ket{1}_{n+1})} =$$
	$$= \frac{1}{C_1}\sum_{i=0}^{2^n-1}{\bc{\ket{i}_{1..n}}}\rc{(\ket{0}_{n+1} - \ket{1}_{n+1})},\; C_1 = 2^{\frac{n+1}{2}}$$
	$$\ket{\psi} = \hat{U}_f\ket{\phi} = \frac{1}{C_1}\sum_{i=0}^{2^n-1}{\bc{\ket{i}_{1..n}}}\rc{(\ket{0 + f(i)}_{n+1} - \ket{1 + f(i)}_{n+1})} =$$
	$$= \frac{1}{C_1}\sum_{i=0}^{2^n-1}{\bc{(-1)^{f(i)}\ket{i}_{1..n}}}\rc{(\ket{0}_{n+1} - \ket{1}_{n+1})}$$
	$$\ket{\psi} = \bc{\ket{\psi}_{1..n}}\rc{\ket{\psi}_{n+1}} = \bc{\ket{\psi}_{1..n}}\rc{\frac{\ket{0}_{n+1} - \ket{1}_{n+1}}{\sqrt{2}}}$$
}
\frame{
	\frametitle{Задача Дойча}
	$$\ket{\theta} = \gc{\hat{H}^{(1..n)}} \ket{\psi}_{1..n} = \frac{1}{2^{\frac{n}{2}}}\sum_{i=0}^{2^n-1}{(-1)^{f(i)}\gc{\hat{H}^{(1..n)}}\ket{i}}$$
	$$\gc{\hat{H}^{(1..n)}}\ket{i} = \frac{1}{2^{\frac{n}{2}}}(\bc{\ket{0..0}} + ...)$$
	$$\ket{\theta} = \rc{\frac{1}{2^n}}((\rc{\sum_{i=0}^{2^n-1}{(-1)^{f(i)}}})\bc{\ket{0..0}} + ...) = \rc{A}\bc{\ket{0..0}} + ...$$
	Если $f$ постоянна, тогда $\rc{A} = 1$, и при измерении $\ket{\theta}$ всегда получим $\bc{\ket{0..0}}$. \newline
	Если $f$ сбалансирована, тогда $\rc{A} = 0$, и при измерении $\ket{\theta}$ никогда не получим $\bc{\ket{0..0}}$.
}

\frame{
	\frametitle{Невозможность квантового копирования}
	$$\hat{U}_{xerox}\ket{\phi}\ket{0} = \ket{\phi}\ket{\phi}$$
	$$\ket{\phi} = c_1\ket{\phi_1} + c_2\ket{\phi_2}$$
	$$\hat{U}_{xerox}(c_1\ket{\phi_1} + c_2\ket{\phi_2})\ket{0} = c_1\ket{\phi_1}\ket{\phi_1} + c_2\ket{\phi_2}\ket{\phi_2} \neq$$
	$$\neq \ket{\phi}\ket{\phi} = c_1^2\ket{\phi_1}\ket{\phi_1} + c_1c_2\ket{\phi_1}\ket{\phi_2} + c_1c_2\ket{\phi_2}\ket{\phi_1} + c_2^2\ket{\phi_2}\ket{\phi_2}$$
}

\newcommand{\ketA}[1]{\rc{\ket{#1}_A}}
\newcommand{\ketB}[1]{\gc{\ket{#1}_B}}
\newcommand{\ketC}[1]{\bc{\ket{#1}_C}}
\newcommand{\ketAB}[1]{\ket{#1}_{\rc{A}\gc{B}}}
\newcommand{\ketBC}[1]{\ket{#1}_{\gc{B}\bc{C}}}
\newcommand{\ketAC}[1]{\ket{#1}_{\rc{A}\bc{C}}}
\newcommand{\ketABC}[1]{\ket{#1}_{\rc{A}\gc{B}\bc{C}}}

\frame{
	\frametitle{Квантовая телепортация}
	$$\ket{\gamma}_C = a\ket{0}_C + b\ket{1}_C$$
	$$\ket{\phi^\pm} = \frac{1}{\sqrt{2}}(\ket{00}\pm\ket{11})$$
	$$\ket{\psi^\pm} = \frac{1}{\sqrt{2}}(\ket{01}\pm\ket{10})$$
	$$\ketABC{\xi} = \ketAB{\phi^+}\ketC{\gamma} = \frac{1}{\sqrt{2}}(\ketAB{00} + \ketAB{11})(a\ketC{0} + b\ketC{1}) =$$
	$$= \frac{1}{2}(\ketAC{\phi^+} + \ketAC{\phi^-}\sigma_z^{\gc{(B)}} + \ketAC{\psi^+}\sigma_x^{\gc{(B)}} + \ketAC{\psi^-}i\sigma_y^{\gc{(B)}}))\ketB{\gamma}$$
	$$
	\sigma_x = \begin{pmatrix}0 & 1 \\ 1 & 0\end{pmatrix},\,
	\sigma_y = \begin{pmatrix}0 &-i \\ i & 0\end{pmatrix},\,
	\sigma_z = \begin{pmatrix}1 & 0 \\ 0 &-1\end{pmatrix}
	$$
}

\end{document}
